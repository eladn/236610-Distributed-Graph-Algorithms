% LIBS
\usepackage[margin=2.5cm, top=2cm, bottom=1.5cm]{geometry}
\usepackage{fancyhdr}
\usepackage{amsmath,amssymb,mathtools,amsthm}
\usepackage{color,soul}
\usepackage{tikz}
\usepackage{csquotes}
\usepackage{listings}
\usepackage{polyglossia,enumitem}
\usepackage{bm}
\usepackage{fontawesome}
\usepackage{chngcntr}


% HEBREW
\setmainlanguage{hebrew}
\newfontfamily\hebrewfont[Script=Hebrew]{David CLM}
\let\hebrewfonttt\ttfamily
\setlist[itemize,1]{label={\fontfamily{cmr}\fontencoding{T1}\selectfont\textbullet}}
\setotherlanguage{english}

% HEADER
\pagestyle{fancy}
\rhead{236610 - אלגוריתמים מבוזרים בגרפים}
\lhead{\texttt{http://bit.ly/CS236610}}

% THEOREMS
\renewcommand*{\proofname}{הוכחה}
\newtheorem{example}{דוגמה}
\counterwithin*{example}{section}
\newtheorem{definition}{הגדרה}
\counterwithin*{definition}{section}
\newtheorem{observation}{אבחנה}
\counterwithin*{observation}{section}
\newtheorem{claim}{טענה}
\counterwithin*{claim}{section}
\newtheorem{lemma}{למה}
\counterwithin*{lemma}{section}
\newtheorem{theorem}{משפט}
\counterwithin*{theorem}{section}
\newtheorem{corollary}{מסקנה}
\counterwithin*{corollary}{section}

% COMMANDS
\newcommand{\defeq}{\vcentcolon=}

% LAYOUT
\setlength\parindent{0pt}
\def\arraystretch{1.5}

% FRONT
\newcommand{\front}[2]{
	\newpage
	\vspace*{1cm}
	\begin{center}
		\Huge{הרצאה \lecnum}
		
		\vspace*{5cm}
		\Huge{#1}
		
		\vspace*{1cm}
		\huge{#2}
	\end{center}
	\newpage
}

% TIKZ
\usetikzlibrary{
	automata
	,arrows
	,arrows.meta
	,shapes
	,positioning
	,matrix
	,decorations.pathmorphing
}

\tikzset{>=latex, thick}
\tikzset{every picture/.style=very thick}

% NODE
\tikzset{default node/.style={
		draw, 
		circle,
		inner sep=1pt,
		minimum size=5mm,
		very thick,
		font=\small,
		black!70,
}}

% LABELS
\tikzset{
	label/.style={
		draw=none
		,sloped
		,rectangle
		,minimum size=0
		,inner sep=0mm
	}
	,label above/.style={
		label
		,midway
		,above=.5mm
	}	
	,label below/.style={
		label
		,midway
		,below=.5mm
	}
	,label inside/.style={
		label	
		,midway
		,fill=white
		,inner sep=2pt
	}
}

\def\padzeros#1{\ifnum#1<10 0\fi\number#1}
\newcommand{\lecture}[1]{
	\def\lecnum{#1}
    \def\paddedlecnum{\padzeros{#1}}
}
\def\insert#1{\input{lec\paddedlecnum/#1}}
