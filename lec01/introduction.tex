\begin{displayquote}
חישוב מבוזר הוא שיטה שבמסגרתה חלקים שונים של משימת חישוב כלשהי המוטלת על תוכנית מחשב מתבצעים במחשבים נפרדים המקושרים ביניהם באמצעות רשת.
\end{displayquote}
\leftline{\textenglish{---} \textit{ויקיפדיה}}

חישוב מבוזר מבוצע ע"י מספר יחידות חישוב שמתקשרות בינן ובין עצמן על מנת לבצע חישוב כלשהו. הביזור מתאפיין בכך שלא תהיה יחידה מרכזית אחת שאחראית על החישוב הגלובאלי.
במהלך הקורס נכיר מודלים שונים של חישוב מבוזר בגרפים, ואת הבעיות הבסיסיות בתחום. נסקור אלגוריתמים וחסמים תחתונים גם יחד. אופי הקורס הינו אלגוריתמי/מתמטי.

\subsection{הגדרת המודל החישובי}
\subsection{מדדי סיבוכיות}
